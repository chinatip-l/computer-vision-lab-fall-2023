\documentclass[12pt,a4paper]{report}

% Packages
\usepackage{geometry}
\usepackage{amsmath}
\usepackage{amsfonts}
\usepackage{amssymb}
\usepackage{graphicx}
\usepackage{listings} % for code listing

% Geometry
\geometry{a4paper, margin=1in}

% Title
\title{Computer Vision Homework Report \\ \Large Homework 1}

\author{Chinatip Lawansuk}
\date{}


% For code listings
\lstset{
  basicstyle=\ttfamily\small,
  breaklines=true,
  frame=single,
  language=C++  % this line sets the language to C++
}

\begin{document}

\maketitle

\tableofcontents

\chapter{Introduction}
This project focuses on develop an algorithm to perform principle operations in Computer Vision from scratch.

\section{Objectives}
\begin{enumerate}
  \item Understand Theory of Computer Vision
  \item Improve C++/Python Programming Skill
\end{enumerate}

\section{Requirements}
\begin{enumerate}
  \item Write a function to convert an image to greyscale image
  \item Write a convolution operation with edge detection kernel using zero padding and stride 1
  \item Write a pooling operation with using Max pooling, 2$\times$2 kernel, and  stride 2
  \item Write a binarisation operation (customise the threshold independently).
\end{enumerate}

\section{System Configuration}
\subsection{Hardware}
\begin{itemize}
  \item CPU\@: Intel Core-i7
  \item GPU\@: NVIDIA
  \item RAM\@: 40 GB
\end{itemize}

\subsection{Software}
\begin{itemize}
  \item OS\@: Windows Subsystem Linux x64 (Ubuntu 22.04.3 LTS Kernel Ver. 5.15.90.1)
  \item GCC Version\@: \verb|11.4.0 x86_64-linux-gnu|
  \item OpenCV Version\@: \verb|4.5.4+dfsg-9ubuntu4|
\end{itemize}

\chapter{Solution, Explanation, and Result}
In this project, explanation is embedded in the comment.
\section{Convert Image to Greyscale Image}
\begin{tabular}{lll}
  Input  & : & Matrix of Input Image  \\
  Output & : & Matrix of Output Image \\
\end{tabular}

\begin{lstlisting}
// offset will be applied for both before, and after the target pixel ----
int out_h = ((img.rows + (2 * padding) - kernel.rows) / stride) + 1;

// create intermediate buffer and add padding, size is w+(2*padding),h+(2*padding)
// then copy the content of image to the buffer
Mat padded;
padded = Mat(img.rows + (2 * padding), img.cols + (2 * padding), img.type(), Scalar(0, 0, 0));
for (int j = 0; j < img.rows; j++)
{
    for (int i = 0; i < img.cols; i++)
    {
        padded.at<Vec3b>(j+padding,i+padding)=img.at<Vec3b>(j,i);
    }
}

\end{lstlisting}


\chapter{Discussion}
Briefly introduce the computer vision tasks you will be performing in this homework.t

\section{Problem Statements}
\subsection{Problem 1}
State the problem statement for Problem 1 here.

\subsection{Problem 2}
State the problem statement for Problem 2 here.kllmm



% Add more problems if needed

\section{Original Images}
\begin{figure}[h]
  \centering

  \caption{Original Image 1}
\end{figure}

\begin{figure}[h]
  \centering

  \caption{Original Image 2}
\end{figure}

\section{Processing Steps and Outputs for Problem 1}
\subsection{Step 1: [Description]}
\subsubsection{Explanation}
Explain what this step does and how it solves or contributes to solving Problem 1.

\subsubsection{Output}
\begin{figure}[h]
  \centering

  \caption{Step 1 Output for Problem 1}
\end{figure}

% Repeat for Steps 2, 3, 4, etc.

\section{Processing Steps and Outputs for Problem 2}
% Similar to the previous section, but for Problem 2

\section{Code Explanation}

\begin{lstlisting}

#include <iostream>
#include <opencv2/opencv.hpp>

int main() {
    // Read the image
    cv::Mat image = cv::imread("original_image1.png");

    // Processing steps for Problem 1
    // ...

    // Processing steps for Problem 2
    // ...

    cv::imwrite("step1_output_problem1.png", image);

    return 0;
}
\end{lstlisting}

\section{Conclusion}
Summarize what you have learned from completing this homework assignment in the context of computer vision.

\appendix

\chapter{Source Code: main.cpp}
\begin{lstlisting}
  
  #include <iostream>
  #include <opencv2/opencv.hpp>
  
  int main() {
      // Read the image
      cv::Mat image = cv::imread("original_image1.png");
  
      // Processing steps for Problem 1
      // ...
  
      // Processing steps for Problem 2
      // ...
  
      cv::imwrite("step1_output_problem1.png", image);
  
      return 0;
  }
  \end{lstlisting}
\chapter{Source Code: func.hpp}
\begin{lstlisting}
  
  #include <iostream>
  #include <opencv2/opencv.hpp>
  
  int main() {
      // Read the image
      cv::Mat image = cv::imread("original_image1.png");
  
      // Processing steps for Problem 1
      // ...
  
      // Processing steps for Problem 2
      // ...
  
      cv::imwrite("step1_output_problem1.png", image);
  
      return 0;
  }
  \end{lstlisting}
\chapter{Source Code: func.cpp}
\begin{lstlisting}
  
  #include <iostream>
  #include <opencv2/opencv.hpp>
  
  int main() {
      // Read the image
      cv::Mat image = cv::imread("original_image1.png");
  
      // Processing steps for Problem 1
      // ...
  
      // Processing steps for Problem 2
      // ...
  
      cv::imwrite("step1_output_problem1.png", image);
  
      return 0;
  }
  \end{lstlisting}

\end{document}
